\documentclass{article}
\usepackage[utf8]{inputenc}
\title{Primer documento}
\author{Roberto Elizarraras Rivera}
\date{Mayo 2021}

\begin{document}
\maketitle
%Esto es sólo un comentario que no se mostrará
Ecuación sencilla
\begin{equation}
F=ma
\end{equation}
Pero si no quiero la m en cursivas
\begin{equation}
F=\mathrm{m}a
\end{equation}
Segundo nivel
\begin{equation}
\sqrt{2gh}
\end{equation}
Como $V=\frac{4}{3}\pi r^3$ el despeje de r es:
\begin{equation}
r=\sqrt[3]{\frac{3V}{4\pi}}
\end{equation}
Tercer nivel
\begin{equation}
v_1=\sqrt{\frac{2(P_1-P_2)}{\rho(\frac{A_1^2}{A_2^2}-1)}}
\end{equation}
Los especiales
\begin{equation}
\sum_{n=1}^{m}n
\end{equation}
Por último una integral
\begin{equation}
\int_{0}^{\infty}\frac{\mathrm{d}x}{\sqrt{x^2-9}}
\end{equation}
\section{2}
\subsection{2.1}
\subsection{2.2}
\subsubsection{2.2.1}
\subsubsection{2.2.2}
\subsection{2.3}
\section{3}
Aquí terminé

\end{document}