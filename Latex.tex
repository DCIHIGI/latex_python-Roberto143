\documentclass{article}
\usepackage[utf8]{inputenc}
\usepackage{multicol}
\usepackage[letterpaper, margin=1in]{geometry}
\usepackage{amsmath}
\usepackage{graphics}
\title{\huge \textbf{Hints on halo evolution in SFDM models with galaxy observations}}
\author{Alma X. González-Morales, Alberto Diez-Tejedor, L. Arturo Ureña-López, and Octavio Valenzuela}
\date{1Instituto de Ciencias Nucleares, Universidad Nacional Autónoma de México,
Circuito Exterior C.U., A.P. 70-543, México D.F. 04510, México
2Departamento de Física, División de Ciencias e Ingenierías,
Campus León, Universidad de Guanajuato, León 37150, México
3Instituto de Astronomía, Universidad Nacional Autónoma de México,
Circuito Exterior C.U., A.P. 70-264, México D.F. 04510, México\\
Dated: October 29, 2018\\ A massive, self-interacting scalar eld has been considered as a possible candidate for the dark
matter in the universe. We present an observational constraint to the model arising from strong
lensing observations in galaxies. The result points to a discrepancy in the properties of scalar eld
dark matter halos for dwarf and lens galaxies, mainly because halo parameters are directly related
to physical quantities in the model. This is an important indication that it becomes necessary to
have a better understanding of halo evolution in scalar eld dark matter models, where the presence
of baryons can play an important role.\\
PACS numbers: 95.30.Sf, 95.35.+d, 98.62.Gq, 98.62.Sb.}
\begin{document}
\maketitle
\begin{multicols}{2}

\section{INTRODUCTION}
The nature of dark matter (DM) remains elusive today,
even though a generic cold particle weakly coupled to the
standard model seems to be the most promising candidate
[\cite{1}]. Treating DM as a bunch of classical particles is
an appropriate efective description for many physical situations.
However, if DM is composed of bosons, the zero
mode can develop a non-vanishing expectation value; this
efect is usually known as Bose-Einstein condensation. A
condensed phase does not admit a description in terms
of classical particles, and the concept of a coherent excitation
(i.e. a classical field) is more appropriate for practical
purposes [\cite{2}]. A specific realization of this scenario
can be provided by the axion [\cite{3}], see also [\cite{4}].\par
In this paper we shall explore the lensing properties
of a generic model of DM particles in a condensate, and
compare the conditions necessary to produce strong lensing
with those required to explain the dynamics of dwarf
galaxies. As a result we will get some insight into halo
evolution arising from this type of models.\par
In particular, we will consider the case of a complex,
massive, self-interacting scalar field \(\phi\) satisfying
the Klein-Gordon (KG) equation, \(\phi- (mc/h)^2-\lambda|\phi|^2\phi=0\), with the box denoting the d'Alembertian
operator in four dimensions. For those natural situations
in which the scalar field mass $m$ is much smaller
than the Planck scale, $m_{Planck}=(hc/G)^{1/2}$, such that
$\vee \equiv \lambda m^2_{Planck}/4\pi m^2 >1$, the coherent (self-gravitating,
spherically symmetric) solutions to the KG equation admit
a very simple expression for the mass density [\cite{5, 6}],
\begin{equation} \label{eq:1}
\begin{split}
\rho(r)=\Bigg\{ \rho_c\frac{\sin(\pi r/r_{max})}{\pi r/r_{max}}\quad \textrm{for} \quad r<r_{max}\\
0 \quad  \quad \quad \quad \textrm{for} \quad                                                          r\geq r_{max}
\end{split}
\end{equation}
As usual we will refer to this model as scalar field dark
matter (SFDM). Here $r_{max} \equiv \sqrt{\pi^2\vee /2} (h/mc)$ is a constant
with dimensions of length (notice that $r_{max}$ is just
the Compton wavelength of the scalar particle, $h/mc$,
scaled by a factor of order $\vee^{1/2}$), and $\rho_c$ the density at
the center of the configuration. The mass density profile
in Eq. (\ref{eq:1}) leads to compact objects of size $r_{max}$, and typical
masses, $4\rho_cr^3_{max}/\pi$, that vary from configuration to
configuration according to the value of the central density.\par
Eq. (\ref{eq:1}) was obtained without taking into account the
gravitational influence of any other matter sources, and
assuming that all the scalar particles are in the condensate.
It has been used as a first order approximation to
describe the distribution of matter in dwarf shperoidals,
which are expected to be DM dominated [\cite{6, 7, 8}]. The mass
distribution would be smooth close to the center of these
galaxies, alleviating the cusp/core problem motivated by
the discrepancies between the observed high resolution
rotation curves and the profiles suggested by N-body simulations
[\cite{9}]; see however [\cite{10}].\par
The dynamics of dwarf galaxies suggests a selfinteracting
scalar with $m^4/\lambda ~ 50 - 75 (eV/c^2)^4$, (i.e.
$r_{max} ~ 5.5 - 7 Kpc$), and typical central densities of the
order of $\rho_c~10^{-3}M./pc^3$, see Ref. [\cite{6}]. We are aware
that Milky Way size galaxies are, at least, an order of
magnitude larger than this value of $r_{max,}$ and then they
do not fit in this model as it stands. Nonetheless, if
not all the DM particles are in the condensate, there is
a possibility to have gravitational configurations where
the inner regions are still described by the mass density
profile in Eq. (\ref{eq:1}), wrapped in a cloud of non-condensed
particles [\cite{11}]. For the purpose of this paper we do not
need to specify the complete halo model. This is because
strong lensing is not very sensitive to the mass distribution
outside the Einstein radius, at most of the order of
a few Kpc, just bellow the expected value of $r_{max}$. We
could not neglect the exterior profile of the halo if we were
interested, for instance, in weak lensing observations.

\section{LENSING PROPERTIES OF SFDM HALOS}
In the weak field limit the gravitational lensing produced
by a mass distribution can be read directly from
the density profile. As usual we assume spherical symmetry,
and use the thin lens approximation, that is, the size
of the object is negligible when compared to the other
length scales in the configuration, i.e. the (angular) distances
between the observer and the lens, $D_{OL}$, the lens
and the source, $D_{LS}$, and from the observer to the source,
$D_{OS}$.\par
Under these assumptions the lens equation takes the
form
\begin{equation} \label{eq:2}
\beta=\theta-\frac{M(\theta)}{\pi D^2_{OL}\theta \Sigma_{cr}},
\end{equation}
with $\beta$ and $\theta$ denoting the actual (unobservable) angular
position of the source, and the apparent (observable) angular
position of the image, respectively, both measured
with respect to the line-of-sight [\cite{12}]. The (projected)
mass enclosed in a circle of radius $\xi$, $M(\xi)$, is defined
from the (projected) surface mass density, $\Sigma(\xi)$, through
\begin{equation} \label{eq:3}
\Sigma(\xi)\equiv \int_{-\infty}^{\infty}dz \rho(z,\xi),\quad M(\xi)\equiv \int_{0}^{\xi}d\xi ' \xi '\Sigma(\xi ').
\end{equation}
Here $\xi = D_{OL}\theta$ is a radial coordinate in the lens plane,
and $z$ a coordinate in the orthogonal direction. Finally
$\Sigma_{cr}\equiv c^2D_{OS}/4\pi GD_{OL}D_{LS}$ is a critical value for the surface
density.\par
In general, Eq. (\ref{eq:2}) will be non-linear in $\theta$, and it could
be possible that for a given position of the source, $\beta$, there
would be multiple solutions (i.e. multiple images) for
the angle $\theta$. This is what happens in the strong lensing
regime to be discussed below. One particular case is that
with a perfect alignment between the source and the lens,
that actually defines the Einstein ring, with an angular
radius of $\theta_E\equiv \theta(\beta=0)$.\par
For a SFDM halo, and in terms of the normalized
lengths $\xi_*\equiv \xi/r_{max}$ and $z_*\equiv z/r_{max}$, the surface mass
density takes the form
\begin{equation} \label{eq:4}
\Sigma_{SFDM}(\xi_*)=\frac{2\rho_cr_{max}}{\pi}\int_{0}^{z_{max}}\frac{\sin(\pi \sqrt{\xi^2_*+z^2_*})}{\sqrt{\xi^2_*+z^2_*}}dz_*,
\end{equation}
with $0\leq \xi_*\leq1$ and $z_{max}=\sqrt{1-\xi^2_*}$. A similar
expression can be obtained for the mass enclosed in a
circle or radius $\xi$, see Eq. (\ref{eq:3}) above. Here we are not
considering the efect of a scalar cloud surrounding the
condensate. For $r\leq r_{max}$ this will appear as a projection
efect, which is usually considered to be small [\cite{13}]. Indeed,
we have corroborated that the inclusion of an outer
isothermal sphere does not afect the conclusions of this
paper.\par
With the use of the expression for the projected mass,
$M_{SFDM}(\xi_*)$, the lens equation simplifies to
\begin{equation} \label{eq:5a}
\beta_*(\theta_*)=\theta_*-\lambda \frac{m(\theta_*)}{\theta_*},
\end{equation}
\begin{figure}
\graphicspath{{E:/Escuela/DCI Física/Herramientas Informáticas y Gestión de la Información}}
\includegraphics{Figura1}
\caption{The lens equation of a SFDM halo model, Eq. (\\ref{eq:5a}), as
a function of $\lambda$. The roots define the Einstein radius, $\theta_{*E}$, and
its local maximum (minimum) the critical impact parameter, 
$\beta_{*cr}$. Both quantities are well defined only for values of $\lambda>\lambda_{cr}\simeq 0.27$, 
which is the threshold value for strong lensing.}
\label{fig:figura1}
\end{figure}
where $m(\xi_*)\equiv M_{SFDM}(\xi_*)/\rho_cr^3_{max}$ is a normalized mass
function, evaluated numerically. Here $\beta_*=D_{OL}\beta/r_{max}$ 
and $\theta_*=D_{OL}\theta/r_{max}$ are the normalized angular positions
of the source and images, respectively, and the parameter $\lambda$ is given by
\begin{equation} \label{eq:5b}
\lambda \equiv \frac{\rho_cr_{max}}{\pi \Sigma_{cr}}=0.57h^{-1}\bigg(\frac{\rho_c}{M.pc^{-3}}\bigg)\bigg(\frac{r_{max}}{kpc}\bigg)\frac{d_{OL}d_{LS}}{d_{OS}}.
\end{equation}
In order to avoid confusion with the self-interaction term, 
$\lambda$, we have introduced a bar in the new parameter $\lambda$. We
have also defined the reduced angular distances $d_A\equiv D_AH_0/c$, and 
considered $H_0\equiv 100h$(km/s)/Mpc as the
Hubble constant today, with $h=0.710\pm0.025$ [\cite{14}].\par
In Fig. \ref{fig:figura1} we show the behavior of the lens equation (\ref{eq:5a})
as the $\lambda$ parameter varies (i.e. for diferent values of the
combination $\rho_cr_{max}$). Some notes are in turn: i) Strong
lensing can be produced only for configurations with $\lambda>\lambda_{cr}\simeq 0.27$, and $ii$) For these 
configurations, only those with an impact parameter$|\beta_*|<\beta_{*cr}$ can produce three
images (note that the actual value of $\beta_{*cr}$ depends on the
parameter $\lambda, \beta_{*cr}(\lambda)$).\par
These conditions on the SFDM profile are very similar
to those obtained for the Burkert model in [\cite{15}]; this is not
surprising because both of them have a core in radius. In
that sense SFDM halos are analogous to those proposed
by Burkert [\cite{16}], but with the advantage that their properties
are clearly connected to physical parameters in the
model.\par
In Fig. \ref{fig:figura2} we show the magnitude of the Einstein radius,
$\theta_{*E}$, as a function of the parameter $\lambda$, where for
comparison we have also plotted the same quantity for
the NFW [\cite{17}] and Burkert [\cite{15}] profiles. The minimum
value of $\lambda$ needed to produce multiple images is higher for
a SFDM halo, $\lambda^{NFW}_{cr}=0<\lambda^{Burkert}_{cr}=2/\pi^2<\lambda^{SFDM}_{cr}\simeq0.27$. 
(Notice that there is an extra factor of $1/4\pi$ in 
our definition of $\lambda$ when compared to that reported in
Ref. [\cite{15}].) SFDM halos seem to require larger values of
\begin{figure}
\graphicspath{{E:/Escuela/DCI Física/Herramientas Informáticas y Gestión de la Información}}
\includegraphics{Figura2}
\caption{The Einstein radius,$\theta_{*E}$, as function of $\lambda^i$, for
SFDM (solid line), NFW (dashed line), and Burkert (dotted
line) halo models. Einstein rings of similar magnitude require
$\lambda^{NFW}<\lambda^{Burkert}<\lambda^{SFDM}$.}
\label{fig:figura2}
\end{figure}
$\lambda$ in order to produce Einstein rings of similar magnitude
to those obtained for the other profiles, but this is in part
due to projection efects that have not been considered
in this paper [\cite{13, 18}].

\section{LENSING VS DYNAMICS}
Taking into account that in SFDM models there is a
critical value for the parameter $\lambda$, $\lambda_{cr}\simeq 0.27$, and considering
the definition in Eq. (\ref{eq:5b}), we can write the condition
to produce strong lensing in the form
\begin{equation} \label{eq:6}
\rho_cr_{max}[M.pc^{-2}]\geq473.68hf_{dist},
\end{equation}
with $f_{dist}\equiv d_{OS}/d_{OL}d_{LS}$ a distance factor.\par
In order to evaluate the right-hand-side (r.h.s.) of
Eq. (\ref{eq:6}), we consider two surveys of multiply-imaged systems,
the CASTLES [\cite{19}] and the SLACS [\cite{20}]. From
them we select only those elements for which the redshifts
of the source and the lens have been determined
(which amounts to approximately 60 elements in each
survey), and calculate the corresponding distance factor
$f_{dist}$ for every element in the reduced sample. In
CASTLES (SLACS) the distance factors are in the interval
$4\leq f_{dist}\leq27, (6\leq f_{dist}\leq 25)$, with a mean value
of $f_{dist}\simeq7$, ($f_{dist}\simeq11$), and then the r.h.s. of Eq. (\ref{eq:6})
takes on values in the range 1400 - 9000, (2000 - 8500).
Some representative elements from SLACS are shown in
Table \ref{TABLE I} (galaxy lensing). In terms of the mean values,
the inequality in Eq. (\ref{eq:6}) translates into
\begin{equation} \label{eq:7a}
\rho_cr_{max}[M.pc^{-2}]\geq 2000, \quad (CASTLES)
\end{equation}
\begin{equation} \label{eq:7b}
\rho_cr_{max}[M.pc^{-2}]\geq 4000, \quad (SLACS)
\end{equation}
These numbers are an order of magnitude greater
than those obtained from dwarf galaxies dynamics,
$\rho_cr_{max}[M.pc^{-2}\simeq100$, when interpreted using the same
density profile [\cite{7}]; see again Table \ref{TABLE I}. This is the main result
of the paper. Remember that the value of $r_{max}$ is related
to the fundamental parameters of the model, which
are the mass of the scalar particle and the self-interaction
term, and it remains constant throughout the formation
of cosmic structure.\par
We must recall that inequalities in Eq. (\ref{eq:7a}) do not take
into account the presence of baryons in galaxies. Gravity
does not distinguish between luminous and dark matter;
then the contribution of the former to the lens could be
significant in some cases. For instance, for those systems
in SLACS the stellar mass fraction within the Einstein
radius is 0:4, on average, with a scatter of 0:1 [\cite{21}].\par
We have corroborated that our estimates in Eq. (\ref{eq:7b}) are
not sensitive to the inclusion of a baryonic component.
To see that we add the contribution of a de Vacouler
surface brightness profile [\cite{22}] to the lens equation,
\begin{equation} \label{eq:8a}
\beta_*(\theta_*)=\theta_*-\lambda \frac{m(\theta_*)}{\theta_*}-\lambda_{lum}\frac{f(\theta_*/r_{e*})}{\theta_*}.
\end{equation}
Here $\lambda_{lum}$ is a parameter analogous to that given in
Eq (\ref{eq:5b}),
\begin{equation} \label{eq:8b}
\lambda_{lum}\equiv \frac{(M/L)L}{2\pi \Sigma_{cr}},
\end{equation}
and $f(x)$ a dimensionless projected stellar mass,
\begin{multline}
f(x)=\frac{1}{2520}[e^q(q^7-7q^6+42q^5-210q^4+840q^3\\-2520q^2+5040q-5040)+5040],
\end{multline}
with $q\equiv -7.76x^{-1/4}$. The mass-to-light ratio, M/L
(from a Chabrier initial mass function), and the efective
radius, $r_e$, for each system in SLACS are reported
in Ref. [\cite{21}]. With the use of Eq. (\ref{eq:8a}) strong lensing is always
possible. Consequently, we must impose a diferent
condition to constrain the product $\rho_cr_{max}$ in each galaxy,
such as demand the formation of Einstein rings of certain
radius.\par
To proceed we use a small subsample of SLACS that includes
configurations with the minimum, maximum, and
mean Einstein radius, and stellar surface mass density,
respectively. This is because the new lens equation is a
function of the ratio $r_{e*}=r_e/r_{max}$; then, to compute the
magnitudes of the Einstein radii, we shall fix the value of
$r_{max}$ a priori.\par
Using the new lens equation we find the value of $\lambda$ that
produces the appropriate Einstein radius for each of the
elements in the subsample. This is done using two different
values of $r_{max}$: 5 and 10 Kpc. The resultant products
$\rho_cr_{max}$ are compatible (in order of magnitude) with the
inequalities obtained from Eq. (\ref{eq:6}). Only for those systems
in the subsample with a high stellar surface mass
density the value of $\rho_cr_{max}$ can decrease substantially,
but it is important to have in mind that all the possible
uncertainties associated to the distribution of the luminous
matter, like the choice of the stellar initial mass
function, will be more relevant in such cases. In general,
these estimations are sensitive to the details of the
particular configurations, and a more exhaustive analysis,
considering the complete sample, will be presented
elsewhere.
\end{multicols}
\begin{table}[h!]
\begin{center}
\begin{tabular}{|c|c|c|c|c|}
\hline
\multicolumn{2}{|c|}{DYNAMICS OF GALAXIES} & \multicolumn{3}{|c|}{GALAXY LENSING}\\
\hline
Galaxy & $\rho_cr_{max}[M.pc^{-2}]$ & Galaxy & $f_{dist}$ & $\rho_cr_{max}[M.pc^{-2}]$\\
\hline
Ho II & 36.19 & J0008-0004 & 6.61 & 2029.68\\
DD0  154 & 66.47 & J1250+0523 & 8.46 & 2832.41\\
DDO 53 & 67.53 & J2341+0000 & 9.12 & 3053.38\\
IC2574 & 81.89 & J1538+5817 & 11.74 & 3930.44\\
NGC2366 & 85.45 & J0216-0813 & 13.03 & 4362.44\\
Ursa Minor & 104.72 & J1106+5228 & 15.74 & 5269.75\\
Ho I & 120.23 & J2321-0939 & 16.23 & 5433.80\\
DD0 39 & 145.94 & J1420+6019 & 19.72 & 6602.26\\
M81 dwB & 265.87 & J0044+0113 & 25.26 & 8457.05\\
\hline
\end{tabular}
\caption{Estimates of the product $\rho_cr_{max}$ for different galaxies. Left. As reported in Refs. [\cite{7}], using galactic dynamics.Right. Derived from equation (\ref{eq:6}) in this paper; recall that these values represent a lower limit (here we show only a representative subsample of the SLACS survey). Note the difference of an order of magnitude between the values of $\rho_cr_{max}$ for dwarf galaxies
in the local universe, and the lower limit of this same quantity for galaxies producing strong lensing at $z~0.$5.}
\label{TABLE I}
\end{center}
\end{table}

\begin{multicols}{2}
\section{DISCUSSION AND FINAL REMARKS}
We have shown that a discrepancy between lensing and
dynamical studies appears if we consider that the SFDM
mass density profile in Eq. (\ref{eq:1}) describes the inner regions
of galactic halos at different redshifts, up to radii of order
5 - 10 Kpc. More specifically, we have found that lens
galaxies at $z~$ 0.5, if correctly described by a SFDM halo
profile, should be denser than dwarf spheroidals in the
local universe, in order to satisfy the conditions necessary
to produce strong lensing.\par
In principle nothing guarantees that halos of different
kind of galaxies share the same physical properties.
Our studies took into account galaxies that are intrinsically
different in terms of their total mass and baryon
concentration. While dwarf galaxies show low stellar surface
brightness, stellar component in massive, early type
galaxies is typically compact and dense.\par
In the standard cosmological model the evolution of
DM halos may trigger differences in concentrations for
halos with different masses due to differences in the assembling
epoch; smaller halos collapsed in an earlier and
denser universe, therefore they are expected to be more
concentrated. However, it is also well known that the
presence of baryons during the assembly of galaxies can
alter the density profile of the host halos and modify
this tendency, making them shallower (supernova feedback
[\cite{23}]), or even cuspier (adiabatic contraction [\cite{24}]).
Therefore, the stellar distribution may reveal different
dynamical evolution for low and high mass halos triggered
by galaxy formation.\par
For SFDM, the dynamical interaction between baryons
and the scalar field may also modify the internal halo
structure predicted by the model, Eq. (\ref{eq:1}), clarifying the
discrepancy. For instance, if the concentration of stellar
distribution were correlated with that of the halo,
like in the adiabatic contraction model when applied to
standard DM halos, this may explain our findings. But
at this time it is unknown how compressible SFDM halos
are, and if such effect will be enough to explain our
results, because there are no predictions on its magnitude.
If the modification triggered by baryons were insuficient, 
then it might be suggesting an intrinsic evolution
of SFDM halos across cosmic time. For example, if big
galaxies emerge as the result of the collision of smaller
ones, then the central densities of the resultant galaxies
would be naturally higher; after all, $r_{max}$ is a constant
in the model, and one would expect that total mass is
preserved in galaxy-galaxy mergers. At this point we do
not know which of these two mechanisms, the intrinsic to
the model, or that due to the evolution of SFDM halos
in the presence of baryons, is the dominant one. In that
sense, a theoretical description of these processes may be
very useful and welcome.\par
A full picture requires a distribution of values for the
central density generated from the evolution of the spectrum
of primordial density perturbations after inflation. 
Such a result is not available now, but it is possible to
start tracing this distribution with galaxy observations.
We present, for the first time, observational constraints
on the dynamical evolution of SFDM halos in the presence
of baryons, that must be considered for future semianalytical/
numerical studies of galaxy formation.

\section{ACKNOWLEDGMENTS}
We are grateful to Juan Barranco for useful comments.
This work was partially supported by PROMEP, DAIPUG,
CAIP-UG, PIFI, I0101/131/07 C-234/07 of the Instituto
Avanzado de Cosmología (IAC) collaboration,
DGAPA-UNAM grant No. IN115311, and CONACyT
México under grants 167335, 182445. AXGM is very
grateful to the members of the Departamento de Física
at Universidad de Guanajuato for their hospitality.\par

\begin{thebibliography}{24}
\bibitem{1}
G. Bertone, D. Hooper, and J. Silk, Phys.Rept.
405, 279 (2005), arXiv:hep-ph/0404175 [hep-
ph]L. Bergstrom(2012), arXiv:1205.4882 [astro-ph.HE]
\bibitem{2}
M. S. Turner, Phys.Rev. D28, 1243 (1983)W. Hu,
R. Barkana, and A. Gruzinov, Phys. Rev. Lett.
85, 1158 (Aug 2000)T. Matos and L. A. Urena-
Lopez, Phys.Rev. D63, 063506 (2001), arXiv:astro-
ph/0006024 [astro-ph]L. A. Urena-Lopez, JCAP 0901,
014 (2009), arXiv:0806.3093 [gr-qc]T. Rindler-Daller
and P. R. Shapiro, Mon.Not.Roy.Astron.Soc. 422, 135
(May 2012), arXiv:1106.1256 [astro-ph.CO]P.-H. Cha-
vanis and T. Harko, Phys.Rev. D86, 064011 (2012),
arXiv:1108.3986 [astro-ph.SR]
\bibitem{3}
R. Peccei and H. R. Quinn, Phys.Rev.Lett. 38, 1440
(1977)P. Sikivie and Q. Yang, ibid. 103, 111301 (2009),
arXiv:0901.1106 [hep-ph]
\bibitem{4}
C. Boehm and P. Fayet, Nucl.Phys. B683, 219 (2004),
arXiv:hep-ph/0305261 [hep-ph]
\bibitem{5}
M. Colpi, S. Shapiro, and I. Wasserman, Phys.Rev.Lett.
57, 2485 (1986)J.-w. Lee and I.-g. Koh, Phys.Rev. D53,
2236 (1996), arXiv:hep-ph/9507385 [hep-ph]
\bibitem{6}
A. Arbey, J. Lesgourgues, and P. Salati, Phys.Rev. D68,
023511 (2003), arXiv:astro-ph/0301533 [astro-ph]
\bibitem{7}
T. Harko, JCAP 1105, 022 (2011), arXiv:1105.2996
[astro-ph.CO]V. Lora, J. Magana, A. Bernal, F. Sanchez-
Salcedo, and E. Grebel, ibid. 1202, 011 (2012),
arXiv:1110.2684 [astro-ph.GA]
\bibitem{8}
V. H. Robles and T. Matos, Mon.Not.Roy.Astron.Soc.
422, 282 (2012), arXiv:1201.3032 [astro-ph.CO]
\bibitem{9}
W. de Blok and A. Bosma, Astron.Astrophys. 385, 816
(2002), arXiv:astro-ph/0201276 [astro-ph]
\bibitem{10}
O. Valenzuela, G. Rhee, A. Klypin, F. Governato,
G. Stinson, et al., Astrophys.J. 657, 773 (2007),
arXiv:astro-ph/0509644 [astro-ph]J. J. Adams, K. Geb-
hardt, G. A. Blanc, M. H. Fabricius, G. J. Hill, et al.,
ibid. 745, 92 (2012), arXiv:1110.5951 [astro-ph.CO]
\bibitem{11}
N. Bilic and H. Nikolic, Nucl.Phys. B590, 575
(2000), arXiv:gr-qc/0006065 [gr-qc]T. Harko and E. J.
Madarassy, JCAP 1201, 020 (2012), arXiv:1110.2829
[astro-ph.GA]
\bibitem{12}
S. Mollerach and E. Roulet, Gravitational lensing and
microlensing, (World Scientic Publishing Company,
204pp, 2002)
\bibitem{13}
T. P. Kling and S. Frittelli, Astrophys.J. 675, 115 (2008),
arXiv:0711.4977 [astro-ph]
\bibitem{14}
N. Jarosik, C. Bennett, J. Dunkley, B. Gold,
M. Greason, et al., Astrophys.J.Suppl. 192, 14 (2011),
arXiv:1001.4744 [astro-ph.CO]
\bibitem{15}
Y. Park and H. C. Ferguson, Astrophys.J. 589, L65
(2003), arXiv:astro-ph/0304317 [astro-ph]
\bibitem{16}
 A. Burkert, Astrophys.J. 447, L25 (Jul. 1995),
arXiv:astro-ph/9504041
\bibitem{17}
C. O. Wright and T. G. Brainerd, Astrophys. J. 534, 34
(May 2000)
\bibitem{18}
E. A. Baltz, P. Marshall, and M. Oguri, JCAP 0901, 015
(2009), arXiv:0705.0682 [astro-ph]
\bibitem{19}
E. Falco, C. Kochanek, J. Lehar, B. McLeod, J. Munoz,
et al.(1999), arXiv:astro-ph/9910025 [astro-ph], http://
www.cfa.harvard.edu/castles/
\bibitem{20}
A. S. Bolton, S. Burles, L. V. Koopmans, T. Treu,
R. Gavazzi, et al., Astrophys.J. 682, 964 (2008),
arXiv:0805.1931 [astro-ph]
\bibitem{21}
M. Auger, T. Treu, A. Bolton, R. Gavazzi, L. Koopmans,
et al., Astrophys.J. 705, 1099 (2009), arXiv:0911.2471
[astro-ph.CO]
\bibitem{22}
D. Maoz and H.-W. Rix, Astrophys.J. 416, 425 (1993)
\bibitem{23}
D. Ceverino and A. Klypin, Astrophys.J. 695, 292 (2009),
arXiv:0712.3285 [astro-ph]F. Governato, C. Brook,
L. Mayer, A. Brooks, G. Rhee, et al., Nature 463, 203
(2010), arXiv:0911.2237 [astro-ph.CO]
\bibitem{24}
G. R. Blumenthal, S. M. Faber, R. Flores, and J. R.
Primack, Astrophys. J. 301, 27 (Feb. 1986)
\end{thebibliography}


\end{multicols}
\end{document}







































